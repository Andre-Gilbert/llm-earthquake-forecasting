% !TEX root =  master.tex
\chapter{Conclusion}\label{ch:conclusion}
To conclude, the project was successfully completed, and its goals were achieved.
We pre-processed the data and trained a robust model, which was then used to build a
comprehensive application. This application features a dashboard displaying recent
earthquakes and an interactive map showing future earthquakes for the next three days
worldwide. Given the extensive volume of the presented data, which might seem overwhelming,
we developed an AI Agent to increase accessibility. This agent allows users to run
personalized forecasts tailored to their specific needs. Since most users are
interested in one or two regions rather than all 25 regions covered by the forecast,
this feature is particularly useful. Additionally, the AI Agent acts as a US Geopolitical Survey (USGS) expert,
providing users with in-depth analysis on potential risks and safety precautions for individual earthquakes.

Recognizing that many people are not tech-savvy and aiming to spread information
about earthquakes to a broad audience, we minimized technical barriers. By integrating
an interactive AI Agent, we made it easier for users to access and understand crucial
earthquake information.

To enhance the predictive accuracy of earthquake forecasting models, several potential
improvements can be considered. Integrating live data such as magma flow, plate tectonics
movements, and real-time seismic activity could provide a more comprehensive and dynamic
understanding of the underlying processes that precede earthquakes. Additionally, exploring
advanced deep learning models, such as \ac{RNNs} and \ac{LSTM}s, which are well-suited for time-series
data, might offer improved prediction capabilities by capturing complex temporal dependencies.
Furthermore, incorporating multi-source data fusion techniques, which combine various types of
geological and environmental data, could enhance the robustness of the model. Continuous
updates and retraining of the model with the latest data, along with cross-disciplinary
collaboration between seismologists and data scientists, would also contribute to refining
the forecasting accuracy.

However, it is important to acknowledge the inherent unpredictability of earthquakes.
Similar to predicting stock market values, earthquakes are fundamentally random and
complex events, making precise forecasting exceptionally difficult. Despite robust
statistical models, accurate predictions remain elusive. Nonetheless, these improvements
could collectively aid in better anticipating seismic events despite their chaotic nature.